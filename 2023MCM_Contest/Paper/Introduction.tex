\section{Introduction}
\subsection{Background}
A web-based word game called \textbf{Wordle} rapidly got  popularity at the beginning of 2022, attracting the players by its neatness, high playability. Although the history of word-guessing game can be traced back to 1955, the age of the Internet gives new birth to it, making Wordle play an important role in both personal recreation and social networks\cite{article1}. With one puzzle a day, players only need to take three minutes every day to enjoy the game and have fun competing with their friends for fewer turns to work it out. Hard Mode is also provided for more professional players who wants to challenge themselves. 
\par Due to various attributes of words, data has shown that people tend to have a different distribution for tries with regard to the solution word. Since the popularity of Wordle also changes with time, the total number scores and scores on Hard Mode vary accordingly.

\subsection{Problem Restatement}
\begin{itemize}
\item[$\bullet$] Design a mathematical model to explain the variation of the number of reported results, and make a prediction on a certain day in the future by applying the model. Find relationship between certain attributes of solution words and the percentage of scores reported that are played in Hard Mode.
\item[$\bullet$] Develop a model to predict the distribution of the reported results for a future date. Make uncertainty analysis of the model and the predictions, and apply it to a real case (for the word EERIE on March 1, 2023). Then, evaluate the degree of confidence of the model we develop.
\item [$\bullet$] Develop a model to tell the solution words apart by difficulty. Then, based on the model, identify the attributes of a given word (e.g. EERIE) and quantify its difficulty of guessing it to match it with the classification we make. Finally, make accuracy analysis of the model.   
\item [$\bullet$] Discuss about other interesting features of the data provided.
\end{itemize}
\subsection{Our Contribution and work}
